\section{Design}

\begin{figure}[t]   %% START_FIGURE
\centerline{\includegraphics[scale=0.3]{./figs/giga-impl-leveldb-clusterfs}}
\caption{
\textbf{High-level design.}
{\small
Our approach for a scalable metadata service integrates two components: a highly 
parallel and load-balanced indexing technique (called \giga{} \cite{GIGA}) to 
partition
metadata over multiple servers and an optimized metadata representation (called
\tfs{} \cite{TableFS}) on each server. 
Our approach aims to layer this integrated solution on top of existing cluster file 
system deployments. 
}
}
\label{fig:design}
\end{figure}       %% END_FIGURE

Figure \ref{fig:design} shows the architecture of our scalable metadata
service that is designed to be layered on existing deployments of cluster file
systems. Our approach uses a client-server architecture and has three components: 
unmodified applications running on clients, the \giga{} directory indexing service 
on clients and servers, and the \tfs{} persistent metadata representation managed 
by the server. 

Using \giga{} and \tfs{} enables us to tackle two key challenges: highly 
concurrent metadata distribution for ingest-intensive parallel applications
such as checkpointing \cite{GIGA} and 
optimized metadata representation that stores all file system
metadata in structured, indexed files manages by existing cluster file system
deployments \cite{tablefs}. 

Remainder of this section describes more details of our approach. 
Section \ref{design.giga} and \ref{design.tablefs} present a primer on how
\giga{} distributes the metadata in a load-balanced manner and how \tfs{}
stores the metadata to speed-up metadata performance on each server. 
Section \ref{design.integration} describes the challenges in effectively
integrating \giga{} and \tfs{} to work with existing cluster file system
deployments. 

\subsection{Scalable partitioning using \giga{}}
\label{design.giga}
%\section*{\giga{} indexing approach}
%\label{indexing}

\giga{} is a distributed hash-based indexing technique that incrementally
divides each directory into multiple partitions that are spread over multiple 
servers \citep{GIGA11}.
Each filename stored in a directory entry is hashed and mapped to a partition 
using an index. 
\giga{} selects a hash partition such that for any distribution of unique filenames, the hash values of 
these filenames will be uniformly distributed in the hash space.
In addition to load-balanced distribution, \giga{} also grows the directory
index incrementally, i.e. all directories start small on a single server, and
then expand to more servers as they grow in size. 

The core idea behind \giga{} is parallel splitting: each server splits 
without system-wide serialization or synchronization.
Every server makes a local decision, without coordinating with other servers, 
about when to split a partition. 
Such uncoordinated growth causes \giga{} servers to have a partial view of the
entire index; there is no central server that holds the global view of the 
partition-to-server mapping.
Each server knows about the partition it stores and the 
identity of another server that knows more about each ``child'' partition resulting
from a prior split by this server. 
This information is known as the per-server split history of its partitions.
The full \giga{} index is a transitive closure of the split history on each
server and represents the lineage of directory partitioning.

The full index (and split history) is also not maintained synchronously by any client.
\giga{} clients can enumerate the partitions of a directory by traversing 
its split histories starting with the first partition that was created during
\texttt{mkdir}.
However, such a full index that is cached by a client may be stale at
any time, particularly for rapidly mutating directories.
\giga{} allows clients to keep using the stale mapping information and
receiving mapping updates from servers. 
More discussion on the cost-benefit of using
inconsistent mapping state is not relevant to this work and can be found in
prior \giga{} literature \citep{GIGA11, GIGA07}. 

%%%%%%%%%%%


\begin{comment}
\textbf{Tolerating inconsistent mapping at clients -- }
%\subsection*{Tolerating inconsistent mapping at clients}
%\label{indexing:inconsistency}
Clients seeking a specific filename find the appropriate partition by probing 
servers, possibly incorrectly, based on their cached index.
To construct this index, a client must have resolved the directory's parent
directory entry which contains a cluster-wide i-node identifying the server and
partition for the zeroth partition $P_0$.
Partition $P_0$ may be the appropriate partition for the sought filename, or it
may not because of a previous partition split that the client has not yet
learned about. 
An ``incorrectly'' addressed server detects the addressing error by recomputing 
the partition identifier by re-hashing the filename.
If this hashed filename does not belong in the partition it has,
this server sends a split history update to the client.
The client updates its cached version of the global index and 
retries the original request.

The drawback of allowing inconsistent indices is that clients may need 
additional probes before addressing requests to the correct server.
The required number of incorrect probes depends on the client request 
rate and the directory mutation rate (rate of splitting partitions).
It is conceivable that a client with an empty index may send O$(log(N_p))$ 
incorrect probes, where $N_p$ is the 
number of partitions, but \giga{}'s split history updates makes this many
incorrect probes unlikely.
Each update sends the split histories of all partitions that reside on a
given server, filling all gaps in the client index known to this server and
causing client indices to catch up quickly.
Moreover, after a directory stops splitting partitions, clients soon after will 
no longer incur any addressing errors.
%\giga{}'s eventual consistency for cached indices is different from LH*'s
%eventual consistency because the latter's idea of independent splitting (called
%pre-splitting) suffers addressing errors even when the index
%stops mutating \citep{lh*:litwin96}. 

\textbf{Key performance insights -- }
Detailed analysis of the scalability and performance of \giga{} studies several
tradeoffs \citep{giga}; the observations that are within the scope of this work
include the load-balancing and incremental growth strategy.


%%%%%%%%%%%
\subsection*{On-line server additions}
\label{indexing.reconfig}

%This section describes how \giga{} adapts to the addition of servers in a 
%running directory service.
%\footnote{Server removal (i.e., decommissioned, not 
%failed and later replaced) is not as
%important for high performance systems so we leave it to be done by user-level
%data copy tools.}

\begin{figure}[t]
\centerline{\includegraphics[scale=0.35]{../common/figures/giga-serveradd}}
\vspace{-10pt}
\caption{\small
\textbf{\giga{} server additions.}
By changing the partition-to-server mapping from round-robin on the original 
server set to sequential on the newly added servers, \giga{} can minimize the amount
of data migrated (shown by arrows indicating splits).
}
\label{fig:giga-adding}
%\vspace{10pt}
%\hrule depth 0.5pt
\end{figure}

When new servers are added to an existing configuration, the system is
immediately no longer load balanced, and it 
should re-balance itself by migrating a minimal number of directory entries
from all existing servers equally. 
Using the round-robin partition-to-server mapping, shown in Figure 
\ref{fig:giga-indexing}, a naive server addition scheme would require 
re-mapping almost all directory entries whenever a new server is added.

\giga{} avoids re-mapping all directory entries on addition of servers by 
differentiating
the partition-to-server mapping for initial directory growth from the mapping
for additional servers.
For additional servers, \giga{} does not use the round-robin partition-to-server
map (shown in Figure \ref{fig:giga-indexing}) and instead 
maps all future partitions to the new servers in a ``sequential manner''.
The benefit of round-robin mapping is faster exploitation of parallelism
when a directory is small and growing, while a sequential mapping for the tail
set of partitions does not disturb previously mapped partitions more than is 
mandatory for load balancing.
Figure \ref{fig:giga-adding} shows an example where the original configuration
has 5 servers with 3 partitions each, and partitions $P_0$ to $P_{14}$ use a 
round-robin rule (for $P_i$, server is $i$ \texttt{mod} $N$, where $N$ is 
number of servers).
After the addition of two servers, the six new partitions $P_{15}$-$P_{20}$
will be mapped to servers using the new mapping rule: $i$ \texttt{div} $M$, 
where $M$ is the number of partitions per server (e.g., 3 partitions/server).

In \giga{} even the number of servers can be stale at servers and clients. 
The arrival of a new server and its order in the global server list is declared
by the cluster file system's configuration management protocol, such as
Zookeeper for HDFS \citep{zookeeper:hunt10}, leading to each existing server
eventually noticing the new server.
Once it knows about new servers, an existing server can inspect its partitions
for those that have sufficient directory entries to warrant splitting and would
split to a newly added server.
The normal \giga{} splitting mechanism kicks in to migrate only directory
entries that belong on the new servers.
The order in which an existing server inspects partitions can be entirely
driven by client references to partitions, biasing migration in favor of active
directories.
Or based on an administrator control, it can also be driven by a background 
traversal of a list of partitions whose size exceeds the splitting threshold. 

\end{comment}


\subsection{Metadata layout using \ldb{}}
\label{design.tablefs}
Our metadata storage backend implements a modified version of \tfs
to manage all the metadata and small files on disk.
\tfs \cite{TableFS} is a stacked file system which uses another file system
as an object store, and organize all metadata and small files in to a single
on-disk table using a Log-Structured Merge (LSM) tree \cite{ONeil1996}.
The reason of using LSM tree is that it buffers new and changed entries in
memory and translate small random disk writes into large sequential writes.
Therefore LSM tree can reduce random disk seeks effectively, and is a natural
fit for metadata intensive workloads. We decribe the structure of LSM tree
and how LSM tree is used in \tfs to store metadata in greater details
in the following sections.

%\begin{figure}[!ht]
\begin{figure}[t]
\includegraphics[scale=0.4]{figs/leveldb}
\caption{
LevelDB is an implementation of LSM tree which represents data on disk
in multiple SSTables that store sorted key-value pairs.
SSTables are grouped into different levels with lower-numbered levels
containing more recently inserted key-value pairs.
Finding a specific pair on disk may search up to all SSTables in level 0
and at most one in each higher-numbered level.
Compaction is the process of combining SSTables
by merge sort and moving combined SSTables into higher-numbered levels.
}
\vspace{10pt}
\hrule
\label{fig:leveldb}
\end{figure}


\textbf{LSM tree and LevelDB --}
\tfs uses an open-source implementation of LSM tree called LevelDB
\cite{LevelDB}. LevelDB provides a simple key-value store interface,
supporting point query and range query. In LevelDB, by default,
a set of changes are spilled to disk when the total size of modified
entries exceeds 4 MB.  When a spill is triggered, called a
minor compaction, the changed entries are sorted, indexed and written to disk
in a format called an SSTable \cite{BigTable}.  These entries may then be
discarded by the in memory buffer and can be reloaded by searching each SSTable
on disk, possibly stopping when the first match occurs if the SSTables are
searched most recent to oldest.  The number of SSTables that need to be
searched can be reduced by maintaining a Bloom filter\cite{bloomfilter} on each,
but, with time, the cost of finding a record not in memory still increases.
Major compaction, or simply ``compaction",
is the process of combining multiple SSTables
into a smaller number of SSTables by merge sort.

As illustrated in Figure \ref{fig:leveldb},
LevelDB extends this simple approach to further
reduce read costs by dividing SSTables into levels.
In 0-th level, each SSTable may contain entries with any key value,
based on what was in memory at the time of its spill.
The higher-numbered levels of LevelDB's SSTables are
the results of compacting SSTables from their own or lower-numbered levels.
In levels excepth the 0-th level, LevelDB maintains the following invariant:
the key range spanning each SSTable is disjoint from
the key range of all other SSTables at that level.
So querying for an entry in the higher levels
only needs read at most one SSTable in each level.
LevelDB also sizes each of the higher levels differentially:
all SSTables have the same maximum size and
the sum of the sizes of all SSTables at level $L$ will not exceed $10^L$ MB.
This ensures that the number of level grows
logarithmically with increasing numbers of entries.

~\\
\textbf{Table schema -- }
The local metadata storage backend aggregates directory entries,
inode attributes and small files into one LSM tree
with a entry for each file and directory.
To link together the hierarchical structure of the user's namespace,
the rows of the table are ordered by a 224-bit key consisting of
the 64-bit inode number of a file's parent directory
and a 160-bit SHA-1 hash value of its filename string (final component of its pathname).
The value of a row contains the file's full name and inode attributes,
such as inode number, ownership, access mode, file size, timestamps (\textit{struct stat} in Linux),
and a symbolic link that contains the actual path of the file object in the object store.
Figure \ref{fig:schema} shows an example of storing a sample file system's metadata into one LevelDB table.

%\begin{figure}[!ht]
\begin{figure}[t]
\centering
\includegraphics[scale=0.4]{figs/schema}
\caption{\normalsize
An example illustrating a table schema for \tfs
to represent metadata and small files as key-value pairs.}
\vspace{10pt}
\hrule
\label{fig:schema}
\end{figure}

All the entries in the same directory have rows that 
share the same first 64 bits in their the table's key.
For $readdir$ operations, once the inode number
of the target directory has been retrieved, 
a scan sequentially lists all entries having 
the directory's inode number as the first 64 bits of their table's key. 
To resolve a single pathname, the metadata server starts searching from the root inode, 
which has a well-known global inode number $(0)$.
Traversing the user's directory tree
involves constructing a search key by concatenating the inode 
number of current directory with the hash of
next component name in the pathname.


\subsection{Integrating \giga{} and \tfs{}}
\label{design.integration}

To effectively integrate the \giga{} distribution mechanism with the
\tfs{} local representation, we tackled several challenges. 

\subsubsection*{Metadata representation.}

\tfs{} stores all metadata in the system including \giga{} hash
partitions for all directories, entries in each hash partition, and other
bootstrapping information such as root entry and \giga{} configuration state.
The general schema used to store all file is shown below in
\texttt{\{key\} --> \{value\}} format:

\begin{verbatim}
{parentDirID,         {attr(dirEntry),
 gigaPartitionID, -->  symlink/data,
 hash(dirEntry)}       gigaMetaState}
\end{verbatim}

The main differences from the \tfs{} schema described in Section
\ref{design.tablefs} is the addition of \texttt{gigaPartitionID} to identify a
\giga{} hash partition and the optional \texttt{gigaMetaState} to store the
hash partition related mapping information. \giga{} related fields are used
only if large directories are distributed over multiple metadata servers.
\footnote{
Since we already store the \texttt{hash} of the directory entry, we can use the
hash-values to identify hash partitions; this optimization will eliminate the
need for \texttt{gigaPartitionID} in the schema.} 

\subsubsection*{Partition splitting.}

Each \giga{} hash partition and its directory entries are stored in sorted \ldb{} 
files in local \tfs{} instance. 
Recall that each \giga{} server process manages its hash partition $P$ that, on 
overflow, is split into another hash partition $P'$ which is stored on a 
different server; this split involves migrating approximately half the entries 
from old partition $P$ to the new hash partition $P'$ on to another server. 
We explored several ways to perform this cross-server partition split.

One approach would be to perform a range scan on partition $P$ and for each
entry in the scan result, check if it needs to be moved to a different server.
All entries that need to be moved to the new partition $P'$ are batched
together and sent in an RPC message to the server that manages partition $P'$.
The recepient server scans the batch and inserts each key in its own \tfs{}
instance. While the simplicity of this approach makes it an attractive
alternative, there is significant cost stemming from \ldb{} compaction
operations: inserts in \ldb{} tables will force merging and rearranging which
degrades performance until it completes.

To minimize the overhead of these background compactions, we extended \ldb{}
used in \tfs{} to support a three-phase split operation. 
First, the split initiator performs a range scan on its \tfs{} instance to find all
entries in the hash-range that needs to be moved to another server. The results
of this scan are written to a separate \ldb{}-format file in a shared storage
volume accessible to all \giga{} servers.
In the second step, the split initiator notifies the split receiver about this new
partition split and the pointer to the new \ldb{}-format file in the shared
storage volume. The split receiver reads this file and performs a bulk
insertion in the \ldb{} table of its \tfs{} instance. This bulk insertion can
directly ``plug in'' the file in the \ldb{} tree structure instead of
iteratively inserting one key at a time.
The final step is a clean-up and commit phase: after the receiver completes the 
bulk insert operation, it empties the shared store volume and notifies the 
initiator, who then deletes the hash-range from its LevelDB instance.
\footnote
{
This three-phase split can be refined even further: \ldb{} can use symbolic links 
to these split files without explicitly copying the files through shared
storage. Because the current release of \ldb{} does not have support for links, we 
left this optimization for future work. 
}

\subsubsection*{Metadata-specific operations.}

\subsubsection*{Accessing file data.}

\subsubsection*{Other challenges.}

