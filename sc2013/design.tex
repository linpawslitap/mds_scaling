\section{Design and implementation}

\begin{figure}[t]   %% START_FIGURE
\centerline{\includegraphics[scale=0.4]{./figs/giga-impl-leveldb-clusterfs}}
\vspace{10pt}
\caption{\textit{
Our scalable metadata service \sys integrates two components: a highly
parallel and load-balanced indexing technique \cite{GIGA11}
to partition metadata over multiple servers,
and an optimized metadata on-disk representation \cite{TableFS} on each server.
This integrated solution is layered on top of existing cluster
file system deployments (e.g. PanFS) to improve metadata
and small file operation efficiency.
}}
\vspace{10pt}
\hrule
\label{fig:design}
\end{figure}       %% END_FIGURE

Figure \ref{fig:design} presents the overall architecture of our scalable
metadata service. Our metadata service is a middleware inserted into
existing deployments of cluster file systems to improve metadata efficiency
while maintaining high I/O bandwidth for data transferring.
The system uses a client-server architecture,
and consists of three core components:

\begin{itemize}
\item{\textbf{Client:}} Applications interact with our middleware
using posix-similar library interface, or the VFS interface exposed
through the FUSE user-level file system \cite{fuse}.
The stateless client-side code redirects applications' file operations
to appropriate destinations according to the types of operations.
All metadata requests (e.g. \texttt{create()} and \texttt{mkdir()}),
and data requests to small files (e.g. \texttt{read()} and \texttt{write()}),
are handled through the metadata indexing modules that address
requests to the appropriate metadata indexing server.
For all data operations to large size files, the client code forwards
them directly to the underlying cluster file system to take full
advantage of data I/O bandwidth.

\item{\textbf{Metadata Indexing Server:}}
Each indexing server manages its local metadata storage backend to store and
access all metadata information and small file data. It uses GIGA+ algorithm to
partition large directory across indexing servers. It also monitors the growth
of small files, and migrates files into the underlying cluster file system
when their sizes exceed the threshold.

\item{\textbf{Metadata Storage Backend:}}
Metadata storage backend is a modified version of \tfs which packs metadata and
small file data into flat files, and stores flat files
in the underlying cluster file system. Since \tfs converts random updates into
sequential writes, it greatly improves metadata performance. In order to
dynamically distribute large directories, metadata storage backend also modifies
\tfs to support exporting and importing flat files in a batch.

\end{itemize}


Using \giga and \tfs enables us to tackle two key challenges: highly
concurrent metadata distribution for ingest-intensive parallel applications
such as check-pointing \cite{PLFS} and
optimized metadata representation that stores all file system
metadata in structured, indexed files managed by existing cluster file system
deployments.

Remainder of this section describes more details of our system.
Section \ref{design.giga} presents a primer on how \giga distributes metadata.
Section \ref{design.tablefs} shows how \tfs stores all file system metadata
and small files using a single on-disk structure on each server.
Section \ref{design.integration} focus on the challenges in effectively
integrating \giga and \tfs to work with existing cluster file systems.


\subsection{Distributed Metadata Indexing}
\label{design.giga}
%\section*{\giga{} indexing approach}
%\label{indexing}

\giga{} is a distributed hash-based indexing technique that incrementally
divides each directory into multiple partitions that are spread over multiple 
servers \citep{GIGA11}.
Each filename stored in a directory entry is hashed and mapped to a partition 
using an index. 
\giga{} selects a hash partition such that for any distribution of unique filenames, the hash values of 
these filenames will be uniformly distributed in the hash space.
In addition to load-balanced distribution, \giga{} also grows the directory
index incrementally, i.e. all directories start small on a single server, and
then expand to more servers as they grow in size. 

The core idea behind \giga{} is parallel splitting: each server splits 
without system-wide serialization or synchronization.
Every server makes a local decision, without coordinating with other servers, 
about when to split a partition. 
Such uncoordinated growth causes \giga{} servers to have a partial view of the
entire index; there is no central server that holds the global view of the 
partition-to-server mapping.
Each server knows about the partition it stores and the 
identity of another server that knows more about each ``child'' partition resulting
from a prior split by this server. 
This information is known as the per-server split history of its partitions.
The full \giga{} index is a transitive closure of the split history on each
server and represents the lineage of directory partitioning.

The full index (and split history) is also not maintained synchronously by any client.
\giga{} clients can enumerate the partitions of a directory by traversing 
its split histories starting with the first partition that was created during
\texttt{mkdir}.
However, such a full index that is cached by a client may be stale at
any time, particularly for rapidly mutating directories.
\giga{} allows clients to keep using the stale mapping information and
receiving mapping updates from servers. 
More discussion on the cost-benefit of using
inconsistent mapping state is not relevant to this work and can be found in
prior \giga{} literature \citep{GIGA11, GIGA07}. 

%%%%%%%%%%%


\begin{comment}
\textbf{Tolerating inconsistent mapping at clients -- }
%\subsection*{Tolerating inconsistent mapping at clients}
%\label{indexing:inconsistency}
Clients seeking a specific filename find the appropriate partition by probing 
servers, possibly incorrectly, based on their cached index.
To construct this index, a client must have resolved the directory's parent
directory entry which contains a cluster-wide i-node identifying the server and
partition for the zeroth partition $P_0$.
Partition $P_0$ may be the appropriate partition for the sought filename, or it
may not because of a previous partition split that the client has not yet
learned about. 
An ``incorrectly'' addressed server detects the addressing error by recomputing 
the partition identifier by re-hashing the filename.
If this hashed filename does not belong in the partition it has,
this server sends a split history update to the client.
The client updates its cached version of the global index and 
retries the original request.

The drawback of allowing inconsistent indices is that clients may need 
additional probes before addressing requests to the correct server.
The required number of incorrect probes depends on the client request 
rate and the directory mutation rate (rate of splitting partitions).
It is conceivable that a client with an empty index may send O$(log(N_p))$ 
incorrect probes, where $N_p$ is the 
number of partitions, but \giga{}'s split history updates makes this many
incorrect probes unlikely.
Each update sends the split histories of all partitions that reside on a
given server, filling all gaps in the client index known to this server and
causing client indices to catch up quickly.
Moreover, after a directory stops splitting partitions, clients soon after will 
no longer incur any addressing errors.
%\giga{}'s eventual consistency for cached indices is different from LH*'s
%eventual consistency because the latter's idea of independent splitting (called
%pre-splitting) suffers addressing errors even when the index
%stops mutating \citep{lh*:litwin96}. 

\textbf{Key performance insights -- }
Detailed analysis of the scalability and performance of \giga{} studies several
tradeoffs \citep{giga}; the observations that are within the scope of this work
include the load-balancing and incremental growth strategy.


%%%%%%%%%%%
\subsection*{On-line server additions}
\label{indexing.reconfig}

%This section describes how \giga{} adapts to the addition of servers in a 
%running directory service.
%\footnote{Server removal (i.e., decommissioned, not 
%failed and later replaced) is not as
%important for high performance systems so we leave it to be done by user-level
%data copy tools.}

\begin{figure}[t]
\centerline{\includegraphics[scale=0.35]{../common/figures/giga-serveradd}}
\vspace{-10pt}
\caption{\small
\textbf{\giga{} server additions.}
By changing the partition-to-server mapping from round-robin on the original 
server set to sequential on the newly added servers, \giga{} can minimize the amount
of data migrated (shown by arrows indicating splits).
}
\label{fig:giga-adding}
%\vspace{10pt}
%\hrule depth 0.5pt
\end{figure}

When new servers are added to an existing configuration, the system is
immediately no longer load balanced, and it 
should re-balance itself by migrating a minimal number of directory entries
from all existing servers equally. 
Using the round-robin partition-to-server mapping, shown in Figure 
\ref{fig:giga-indexing}, a naive server addition scheme would require 
re-mapping almost all directory entries whenever a new server is added.

\giga{} avoids re-mapping all directory entries on addition of servers by 
differentiating
the partition-to-server mapping for initial directory growth from the mapping
for additional servers.
For additional servers, \giga{} does not use the round-robin partition-to-server
map (shown in Figure \ref{fig:giga-indexing}) and instead 
maps all future partitions to the new servers in a ``sequential manner''.
The benefit of round-robin mapping is faster exploitation of parallelism
when a directory is small and growing, while a sequential mapping for the tail
set of partitions does not disturb previously mapped partitions more than is 
mandatory for load balancing.
Figure \ref{fig:giga-adding} shows an example where the original configuration
has 5 servers with 3 partitions each, and partitions $P_0$ to $P_{14}$ use a 
round-robin rule (for $P_i$, server is $i$ \texttt{mod} $N$, where $N$ is 
number of servers).
After the addition of two servers, the six new partitions $P_{15}$-$P_{20}$
will be mapped to servers using the new mapping rule: $i$ \texttt{div} $M$, 
where $M$ is the number of partitions per server (e.g., 3 partitions/server).

In \giga{} even the number of servers can be stale at servers and clients. 
The arrival of a new server and its order in the global server list is declared
by the cluster file system's configuration management protocol, such as
Zookeeper for HDFS \citep{zookeeper:hunt10}, leading to each existing server
eventually noticing the new server.
Once it knows about new servers, an existing server can inspect its partitions
for those that have sufficient directory entries to warrant splitting and would
split to a newly added server.
The normal \giga{} splitting mechanism kicks in to migrate only directory
entries that belong on the new servers.
The order in which an existing server inspects partitions can be entirely
driven by client references to partitions, biasing migration in favor of active
directories.
Or based on an administrator control, it can also be driven by a background 
traversal of a list of partitions whose size exceeds the splitting threshold. 

\end{comment}


\subsection{Metadata Storage Backend}
\label{design.tablefs}
Our metadata storage backend implements a modified version of \tfs
to manage all the metadata and small files on disk.
\tfs \cite{TableFS} is a stacked file system which uses another file system
as an object store, and organize all metadata and small files in to a single
on-disk table using a Log-Structured Merge (LSM) tree \cite{ONeil1996}.
The reason of using LSM tree is that it buffers new and changed entries in
memory and translate small random disk writes into large sequential writes.
Therefore LSM tree can reduce random disk seeks effectively, and is a natural
fit for metadata intensive workloads. We decribe the structure of LSM tree
and how LSM tree is used in \tfs to store metadata in greater details
in the following sections.

%\begin{figure}[!ht]
\begin{figure}[t]
\includegraphics[scale=0.4]{figs/leveldb}
\caption{
LevelDB is an implementation of LSM tree which represents data on disk
in multiple SSTables that store sorted key-value pairs.
SSTables are grouped into different levels with lower-numbered levels
containing more recently inserted key-value pairs.
Finding a specific pair on disk may search up to all SSTables in level 0
and at most one in each higher-numbered level.
Compaction is the process of combining SSTables
by merge sort and moving combined SSTables into higher-numbered levels.
}
\vspace{10pt}
\hrule
\label{fig:leveldb}
\end{figure}


\textbf{LSM tree and LevelDB --}
\tfs uses an open-source implementation of LSM tree called LevelDB
\cite{LevelDB}. LevelDB provides a simple key-value store interface,
supporting point query and range query. In LevelDB, by default,
a set of changes are spilled to disk when the total size of modified
entries exceeds 4 MB.  When a spill is triggered, called a
minor compaction, the changed entries are sorted, indexed and written to disk
in a format called an SSTable \cite{BigTable}.  These entries may then be
discarded by the in memory buffer and can be reloaded by searching each SSTable
on disk, possibly stopping when the first match occurs if the SSTables are
searched most recent to oldest.  The number of SSTables that need to be
searched can be reduced by maintaining a Bloom filter\cite{bloomfilter} on each,
but, with time, the cost of finding a record not in memory still increases.
Major compaction, or simply ``compaction",
is the process of combining multiple SSTables
into a smaller number of SSTables by merge sort.

As illustrated in Figure \ref{fig:leveldb},
LevelDB extends this simple approach to further
reduce read costs by dividing SSTables into levels.
In 0-th level, each SSTable may contain entries with any key value,
based on what was in memory at the time of its spill.
The higher-numbered levels of LevelDB's SSTables are
the results of compacting SSTables from their own or lower-numbered levels.
In levels excepth the 0-th level, LevelDB maintains the following invariant:
the key range spanning each SSTable is disjoint from
the key range of all other SSTables at that level.
So querying for an entry in the higher levels
only needs read at most one SSTable in each level.
LevelDB also sizes each of the higher levels differentially:
all SSTables have the same maximum size and
the sum of the sizes of all SSTables at level $L$ will not exceed $10^L$ MB.
This ensures that the number of level grows
logarithmically with increasing numbers of entries.

~\\
\textbf{Table schema -- }
The local metadata storage backend aggregates directory entries,
inode attributes and small files into one LSM tree
with a entry for each file and directory.
To link together the hierarchical structure of the user's namespace,
the rows of the table are ordered by a 224-bit key consisting of
the 64-bit inode number of a file's parent directory
and a 160-bit SHA-1 hash value of its filename string (final component of its pathname).
The value of a row contains the file's full name and inode attributes,
such as inode number, ownership, access mode, file size, timestamps (\textit{struct stat} in Linux),
and a symbolic link that contains the actual path of the file object in the object store.
Figure \ref{fig:schema} shows an example of storing a sample file system's metadata into one LevelDB table.

%\begin{figure}[!ht]
\begin{figure}[t]
\centering
\includegraphics[scale=0.4]{figs/schema}
\caption{\normalsize
An example illustrating a table schema for \tfs
to represent metadata and small files as key-value pairs.}
\vspace{10pt}
\hrule
\label{fig:schema}
\end{figure}

All the entries in the same directory have rows that 
share the same first 64 bits in their the table's key.
For $readdir$ operations, once the inode number
of the target directory has been retrieved, 
a scan sequentially lists all entries having 
the directory's inode number as the first 64 bits of their table's key. 
To resolve a single pathname, the metadata server starts searching from the root inode, 
which has a well-known global inode number $(0)$.
Traversing the user's directory tree
involves constructing a search key by concatenating the inode 
number of current directory with the hash of
next component name in the pathname.


\subsection{Integrating \giga{} and \tfs{}}
\label{design.integration}

To effectively build a middleware that integrates
the \giga distribution mechanism and \tfs,
we have to tackle two main challenges:
one is to modify \tfs to support for
migrating directory partition required by \giga;
the other is to decouple metadata and data paths to
achieve high performance for both paths.
This section discusses the modifications we made to
overcome the two challenges.


~\\
\textbf{Metadata representation -- }
\tfs stores all metadata including \giga hash
partitions for directories, entries in each hash partition, and other
bootstrapping information such as root entry and \giga configuration state.
The general schema used to store all file is:

\begin{table}[!htc]
\begin{tabular}{c|c}
key & \texttt{parentDirID,gigaPartitionID,hash(dirEntry)} \\
\midrule \\
value & \texttt{attr(dirEntry),[symlink|data|gigaMetaState]} \\
\end{tabular}
\label{tab:keyschema}
\end{table}

The main difference from the \tfs schema described in Section
\ref{design.tablefs} is the addition of two \giga specific fields:
\texttt{gigaPartitionID} to identify a
\giga hash partition and \texttt{gigaMetaState} to store the
hash partition related mapping information for directories.
These \giga related fields are used only
if large directories are distributed over multiple metadata servers.
A optimization is to eliminate \texttt{gigaPartitionID} in the key
by using the same hash function for both \giga and \tfs keys,
since the hash of entry name can determine the partition ID.

~\\
\textbf{Partition splitting -- }
The local \tfs instance stores \giga hash partitions and their directory
entries as SSTable Files in the underlying cluster file system.
Recall that each \giga server process splits a hash partition $P$ on
overflow and creates another hash partition $P'$ which is managed by a
different server; this split involves migrating approximately half the entries
from old partition $P$ to the new hash partition $P'$ on another server.
During splitting, the partition in migration has to be locked from
concurrent accessing for correctness.
We explored several ways to perform this cross-server partition split.

A straightforward solution would be to perform a range scan on
partition $P$, and remove about half the entries
(that will be migrated to the new partition $P'$) from $P$.
All removed entries are batched together
and sent in an RPC message to the server that will manage partition $P'$.
The split receiver inserts each key in the batch into its own \tfs instance.
While simplicity of this solution makes it attractive,
it is slow in practice and vulnerable to failures during splitting.
We have to devise a faster and safer technique to
reduce the time that the splitting range is in lock.

The immutability of SSTables in \ldb makes such a fast bulk insert possible --
an SSTable can be added to Level 0 without its data being pushed through the
write-ahead log and minor compaction process.
To take advantage of this opportunity, we extended \tfs
to support a three-phase split operation:

\begin{itemize}
\item{Phase 1:} The split initiator performs a range scan on its \tfs instance
to find all entries in the hash-range that needs to be moved to another server.
Instead of packed into an RPC message,
the results of this scan are written in SSTable format to files in the
underlying cluster file system.

\item{Phase 2:} The split initiator notifies the split receiver about
the paths to the SSTable-format files in a much smaller RPC message.
Since these files are stored in shared storage,
the split receiver directly inserts these files as symbolic links
into the \ldb tree structure without actually copying these files.

\item{Phase 3:} The final step is a clean-up and commit phase:
after the receiver completes the bulk insert operation, it notifies the
initiator, who then deletes the migrated hash-range from its \tfs instance
and unlocks the range.
\end{itemize}

The three phases of splitting can be refined even further:
Assume that the splitting is initiated at the time $T_{split}$.
The split initiator can generate SSTables containing entries
older than $T_{split}$ without locking the hash range.
When the generation of SSTables with entries older than $T_{split}$ is finished,
the split initiator can lock the hash range and then write SSTables with
newly added or updated entries later than $T_{split}$.
By doing so, the duration of locking splitting hash range can be further reduced.
However, due to the code complexity of this optimization,
we left this optimization for future work.

~\\
\textbf{Decoupled data and metadata path -- }
All metadata and small file operations go through the \giga server;
however, following the same path for data operations on large files
would incur an unnecessary performance penalty of shipping data over the network on extra time.
This penalty can be significant in HPC use-cases
where large files can easily be gigabytes to terabytes in size.

To avoid this penalty our middleware is designed to perform all
data-path operations of large files directly
through the cluster file system module in client machine.
Figure \ref{fig:design} illustrates this data path (in BLUE color).
Once the application tries to open a file with size greater than $T$,
our client library code will get back a symbolic link to the physical
path in the cluster file system, and open it locally.
All subsequent accesses to this large file will force
the client operating system to read/write through the native cluster file system client.
Thus applications can achieve the same data bandwidth.
FUSE-based clients can use the same trick but with additional overhead
of double context switching and memory copying \cite{PLFS}.
Recent work \cite{fuseopt} shows that FUSE kernel module can be modified
to make performance degradation less than $3\%$. We have not implemented
this modification in our current prototype.

While the file is open, some of its attributes (e.g., file size and last access time)
may change relative to \tfs's per-open copy of the attributes.
\giga servers monitor what large files are currently opened for access.
For attribute quires to these open files, \giga will directly query the underlying
cluster file system to get the most updated attributes.
Later \giga will synchronize these changes on file close on the metadata path.
Other attribute changes relative to permissions can be updated on-flight
through \giga servers.
By doing so, \giga servers can achieve the same level of metadata consistency
as provided by the underlying cluster file system.


\textbf{Layering on Panasas file system -- }
In PanFS \cite{PanFS}, \textit{Volume} is the basic logical unit
for administrators to manage the storage pool of PanFS.
The volume is a directory hierarchy with quota limit, and appears
as a directory below the single mount point of the whole storage system.
Each volume is only managed by a single metadata manager, but
the data of files created in the volume are spread over the entire storage pool.

A typical PanFS storage cluster may have multiple shelves,
each shelf consisting of one or two metadata managers and 9 or 10 storage nodes.
However, since a volume is assigned to a particular metadata manager,
all the metadata accesses to any files/directories in a volume
can only be served by that single metadata manager.

To take advantage of multiple metadata managers in PanFS,
our middleware load balance the storage of SSTables and large files across volumes.
Assume the case that we run the same number of GIGA+ server processes
as the available metadata managers in PanFS.
GIGA+ server processes can be run in the same machines that run PanFS metadata managers,
or in other machines behaving like proxies.
For each metadata manager, we create a volume and assign the volume
to a unique GIGA+ server process.
The GIGA+ server process then stores all the SSTables,
and large files in its managed directory partitions to it assigned volume.

\textbf{Fault Tolerance Design -- }

