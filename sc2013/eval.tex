\section{Experimental Evaluation}

Using metadata-intensive and data-intensive benchmarks,
we evaluate our middleware's overall performance
and explore how its system design contribute to meeting its goals.
We specifically examine: (1) its improvement on the scalability and
efficiency of metadata service for cluster file systems,
especially in concurrent file creation workloads;
(2) the data throughput in N-N checkpointing workloads
when layering our middleware on top of PanFS.

\textbf{Implementation}
Our prototype is implemented in 14K lines of C using a modular design.
\tfs, \ldb and PanFS are modular plugins that can be replaced by other backends
that follow the API semantics. The current version implements most posix
file system operations except \texttt{hardlink}, \texttt{rename},
and operations for extended attributes.

\textbf{Evaluation System }
We evaluate the overall performance of our prototype on
a 5-shelves PanFS cluster with five 64-core test nodes. The detailed hardware
and software configuration is shown in Table \ref{tab:setting}.
We also used another cluster to demonstrate that our middleware solution
can be layered on other file system deployments as well.
The second cluster setting emulates a case that each GIGA+ server runs in
a different NFS node and manages its own \tfs instance locally
to scale the metadata performance of NFS.
In all tests, the client uses library version code;
the threshold for splitting a partition is always 8,000 entries;
and \tfs managed by GIGA+ server syncs its data every 5 seconds.

\begin{table}
\begin{tabular}{lcc}
\toprule
      & Cluster 1 & Cluster 2 \\
\midrule
\#Nodes & 5 & 64 \\
\hline
OS &   CentOS 6.3 &  Ubuntu 12.10 \\
Kernel & 2.6.32 x86\_64 & 3.6.6 x86\_64 \\
\hline
CPU & AMD Opteron 6272 &  AMD Opteron 242 \\
    & 64 Cores & Dual Core\\
\hline
Memory & 128GB DDR &  16GB DDR \\
\hline
Network &       &           \\
\hline
Storage & PanFS & Western Digital \\
System &      5 Shelves & Local hard disk  \\
       &   5MDS + 50 ODS &  2TB per node  \\
& & 100 seeks/sec \\
& & rand. seeks   \\
& & 137.6 MB/sec  \\
& & seq. reads    \\
& & 135.4 MB/sec  \\
& & seq. writes   \\
\bottomrule \\
\end{tabular}
\caption{
\textit{The settings of two clusters for evaluation.}
}
\label{tab:setting}
\end{table}

In the following sections, we will first show the evaluation
on the end-to-end performance of the integrated system on top of PanFS,
and then present the results of a strong scaling experiment on another platform. 

\subsection{Full System Benchmark}
\label{sec:fullsystem}
We performed an end-to-end evaluation of our prototype in our second cluster.
This cluster has 5 test nodes, and is connected to
a 5-shelves PanFS storage cluster with 5 metadata nodes and 50 storage nodes.
Each test node has 64 cores and is using 40GE NIC, which are enough
to saturate the data bandwidth of our PanFS storage cluster.
Because of some technique difficulties,
we did not run our GIGA+ server processes inside the metadata node.
Instead, we co-locate our GIGA+ server processes
with client processes in the test nodes.
Each test node runs a GIGA+ server that is assigned to a metadata node
as explained in Section \ref{design.integration}.
We ran a series of HPC benchmark runs using the open source \textit{mdtest}
synthetic benchmark \cite{mdtest}
and File System Test Suite checkpoint benchmark from LANL \cite{mpiio}
to test metadata path and data path separately.

\subsubsection*{Metadata Intensive Workloads}
To evaluate the metadata performance of layering on PanFS,
we use mdtest to generate a three-step workload:
The first step, similar to the last one, is to create 5 million
zero-files in a single directory;
the second step is to perform a $stat()$ on random files in the directory;
the third step is to delete all the files in the directory in a random order.
Each step involves multiple clients to issue the operations concurrently.

If we directly use the above workload to directly compare our layered system
against the original PanFS, it would not be fair enough.
This is because a single directory can only use the hardware resource
of one metadata manager in PanFS,
and PanFS also limits a single directory to 1 million files.
Therefore we chose to compare native PanFS creating 1 million files
in 5 different directories owned by 5 different metadata managers.
The total number of clients used for testing the two systems
are kept the same.


\begin{figure}[t]  %%%%%%%%%%%%%%%%%%%%%%%
\centerline{\includegraphics[scale=0.6]{./figs/zero_file_creation_on_panfs}}
\vspace{10pt}
\caption{\normalsize
\textit{Create five million zero-length files in one empty directory
with different number of clients.}
}
\vspace{10pt}
\hrule
\label{graph:creation_clients}
\end{figure}       %%%%%%%%%%%%%%%%%%%%%%%

Figure \ref{graph:creation_clients}

\begin{figure}[t]  %%%%%%%%%%%%%%%%%%%%%%%
\centerline{\includegraphics[scale=0.6]{./figs/mdtest}}
\vspace{10pt}
\caption{\normalsize
\textit{mdtest:
The average throughput of different operations in mdtest
when generating 5 million zero-length files in a single shared directory.
Since PanFS has a hard limit to allow only create 1 million entries
in one directory, the bar showing PanFS with 1 volume only gives
the average throughput for the case of creating 1 million entries.
}
}
\vspace{10pt}
\hrule
\label{graph:mdtest_ops}
\end{figure}       %%%%%%%%%%%%%%%%%%%%%%%


\begin{figure}[t]  %%%%%%%%%%%%%%%%%%%%%%%
\centerline{\includegraphics[scale=0.5]{./figs/small_file_creates}}
\vspace{10pt}
\caption{\normalsize
\textit{Create 5 million small files with different size
in one share directory}
}
\vspace{10pt}
\hrule
\label{graph:smallfiles}
\end{figure}       %%%%%%%%%%%%%%%%%%%%%%%

\textbf{Data Intensive Workloads}
%Library version not FUSE
%Clean cache

The LANL filesystem checkpoint benchmark can
generate many types for HPC checkpoint I/O patterns.
For all of our tests, we configured the benchmark to
generate a concurrently written N-N checkpoint.
All checkpoint file I/O is performed by a set of processes
that synchronize with each other using MPI barriers.
In the first phase of the benchmark each process opens the
freshly created checkpoint file for writing and
then waits at a barrier until all nodes are ready to write.
Once all nodes are ready, each node starts
concurrently writing the checkpoint data to its own file.
Each node continues writing to the checkpoint file
until it has written the specified number
of access units, then it waits at an MPI barrier
until all the other nodes have completed writing the data.
Once writing is complete and an MPI barrier reached,
each node syncs its data to disk, closes the file, and then
waits at a file barrier before finishing.
Before starting the read phase we terminate all processes
accessing the underlying file so that
we can unmount the filesystem in order to ensure that
all freshly written data has been flushed from all the nodes'
memory to avoid cached data from unfairly biasing our read performance.
After the filesystem has been mounted and restarted,
the benchmark reads the checkpoint in the same way it was written,
however we shift, so each process will read
the file generated by another process.

\begin{figure}[t]  %%%%%%%%%%%%%%%%%%%%%%%
\centerline{\includegraphics[scale=0.6]{./figs/checkpointing_write}}
\vspace{10pt}
\caption{\normalsize
\textit{
The aggregated write throughput in N-N check-pointing workload.
Each volume receives 640 GB data.
}
}
\vspace{10pt}
\hrule
\label{graph:checkpoint_write}
\end{figure}       %%%%%%%%%%%%%%%%%%%%%%%

\begin{figure}[t]  %%%%%%%%%%%%%%%%%%%%%%%
\centerline{\includegraphics[scale=0.6]{./figs/checkpointing_read}}
\vspace{10pt}
\caption{\normalsize
\textit{
The aggregated read throughput in N-N check-pointing workload.
}
}
\vspace{10pt}
\hrule
\label{graph:checkpoint_read}
\end{figure}       %%%%%%%%%%%%%%%%%%%%%%%



\subsection{Strong Scaling Experiment}
Next, we evaluated the scalability of our distributed metadata middleware prototype.
The scalability experiment was conducted in another 64-nodes cluster.
Each node had a GIGA+ indexing server process that manages its own \tfs
instance whose SSTables are stored on a local disk running Ext4 file system.
To emulate shared storage for split operations,
we used a NFS-mounted directory accessible from all machines;
this shared directory was only used for moving SStables of
splitting directory partitions across servers.

The workload we used is a \textit{strong scaling} experiment, i.e.
creating 1 million files per server, for a total of 64 million files in the
64-server configuration. We varies the number of servers from 1 to 64
to see how the performance of the integration scale with more machine power.

Figure \ref{graph:ldb-scaling} shows the instantaneous throughput
during the concurrent create workload.
The main result in this figure is that as the number of servers doubles the
throughput of the system also scales up. With 64 servers, \giga can achieve a
peak throughput of about 190,000 file creates per second.
The prototype delivers peak performance after the directory workload
has been spread among all servers.
Reaching steady-state, the throughput quickly grows
due to the splitting policies adopted by \giga.

After reaching the steady state, throughput slowly drops
as \tfs builds a larger metadata store.
In fact, in large setups with 8 or more servers,
the peak throughput drops by as much as 25\% (in case of the 64-server setup).
This is because when there are more entries already existing in \tfs,
it requires more compaction work to maintain invariants inside \ldb
and to perform a negative lookup before each create
has to search more SSTables on disk.
In theory, the work of inserting a new entry to a LSM-tree is $O(\log_{B}(n))$
where $n$ is the total number of inserted entries, and $B$ is a constant factor
proportional to the average number of entries transferred in each disk request
\cite{Bender2007}.
Thus we can use the formula $\frac{a\cdot S+b}{\log{T}}$ to
approximate the throughput timeline in Figure \ref{graph:ldb-scaling},
where $S$ is the number of servers, $T$ is the running time,
and $a$ as well as $b$ are constant factors
relative to the disk speed and splitting overhead.
This estimation projects that when inserting 64 billion files with 64 servers,
the system may deliver an average of 1,000 operations per second per server,
i.e. 64,000 operations per second in aggregate.

\begin{figure*}[t]
\centerline{\includegraphics[scale=0.65]{./figs/ldb_insertrate}}
\vspace{10pt}
\caption{\normalsize
\textit{Our middleware metadata service prototype shows promising scalability
up to 64 servers.
Note that at the end of the experiment,
the throughput drops to zero
because clients stop creating files as they finish 1 million files per client.
And the solid lines in each configuration are Bezier
curves to smooth the variability.}
}
\vspace{10pt}
\hrule
\label{graph:ldb-scaling}
\end{figure*}


