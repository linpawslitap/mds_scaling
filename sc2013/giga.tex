\giga is a distributed hash-based indexing technique that incrementally
divides each directory into multiple partitions that are spread over multiple
servers \cite{GIGA11}.
Each filename stored in a directory entry is hashed and mapped to a partition
using a per-directory index.
\giga selects a hash partition such that for any distribution of unique filenames,
the hash values of these filenames will be uniformly distributed in the hash space.
This makes load balancing much easier.
In addition to load-balanced distribution, \giga also grows the directory
index incrementally, i.e. all directories start small on a single server, and
then expand to more servers as they grow in number of entries.

A core scalability idea in \giga is parallel splitting: each server splits
without system-wide serialization or synchronization.
Such uncoordinated growth causes \giga servers to have only a partial view of
the entire per-directory index;
there is no central server that holds the global view of the
partition-to-server mapping.
Each server knows about the partitions it stores and knows the
identity of other server that know more about each ``child'' partition
resulting from a prior split by this server.
This information is known as the per-server split history of
a directory's partitions.
The full per-directory \giga index is
a transitive closure of the split history on each
server and represents the lineage of the directory's partitioning.

The full index (and split history) of a directory
is also not maintained synchronously by any client.
\giga clients can search through the partitions of a directory by traversing
its split histories starting with the first partition that was created during
\texttt{mkdir} and clients can cache what they learn.
However, such an opportunistically cached by a client may be
incomplete or stale at any time, particularly for rapidly mutating directories.
\giga allows clients to keep using stale mapping information
because addressed servers verify and brief to clients as needed.
More discussion of the cost-benefit of using
inconsistent mapping state is not relevant to this work and can be found in
prior \giga{} literature \cite{GIGA07, GIGA11}.
