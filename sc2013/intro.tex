\section{Introduction}
Lack of a highly scalable and parallel metadata service is the
Achilles heel for many cluster file system deployments in both the HPC world
\cite{hpcs-io:2008, hecfsio:tr06} and the Internet services world \cite{HDFS}.
This is because most cluster file systems have used centralized, single-node
metadata management, and focused merely on scaling the data path,
i.e. providing high bandwidth parallel I/O to files that are gigabytes in size.


This inherent metadata scalability handicap limits numerous massively parallel
applications that produce workloads
requiring concurrent and high-performance metadata operations.
One such example, file-per-process checkpointing, requires the metadata service to
handle large number of file creates and updates at very high speeds \cite{PLFS}.
Another example, storage management, produces read-intensive metadata workload
that typically scans the metadata of the entire file system to perform
administration tasks for analyzing and querying metadata \cite{filemgmt-ucsc, magellan-ucsc}.
Thirdly, even in the era of big data,
most things in many cluster file systems are small \cite{Dayal, brent13}.
Scalable systems should expect the numbers of small files
to soon achieve and exceed billions,
a known challenge for many existing cluster file systems \cite{GIGA11}.

We envision a scalable metadata service with two goals.
The first goal -- \textit{evolution, not revolution} -- emphasizes the need for
a solution that adds new support to existing cluster file systems that lack a
scalable metadata path.
Although newer cluster file systems, including Google's Colossus file system
\cite{50mfiles-in-googlefs:fikes10}, OrangeFS \cite{OrangeFS}, UCSC's Ceph \cite{ceph:weil06} and 
Copernicus \cite{sfs-ucsc}, promise a distributed metadata
service, it is
undesirable to replace existing cluster file systems running in large production
environments just because their metadata path does not provide the desired
scalability or the desired functionality.
Several large cluster file system installations, such as Panasas PanFS running
at LANL \cite{panfs:welch08} and PVFS running on Argonne BG/P
\cite{bgp, pvfs:www}, can benefit from a solution that provides,
for instance, distributed directory support
that does not require any modifications to the running cluster file system.
The second goal -- \textit{generality and de-specialization} -- promises a
fully, distributed and
scalable metadata service that performs well for ingest, lookups, and scans.
In particular, all metadata, including directory entries, i-nodes and block
management, should be stored in one structure; this is different from
today's file systems that use specialized on-disk structures for each type of
metadata.

To realize these goals, this paper makes a case for a scalable metadata service
middleware that layers on existing cluster file system deployments and
distributes file system metadata, including the namespace tree, small
directories and large directories, across many servers.
Our key idea is to effectively synthesize a concurrent indexing
technique to distribute metadata with a tabular, on-disk representation of all
file system metadata.

For distributed indexing, we re-use the concurrent, incremental, hash-based
\giga indexing technique \cite{GIGA11}.
The main shortcoming of the \giga prototype is that splitting
the metadata partitions for better load-balancing involves migrating the
directory entries and the associated file data \cite{GIGA11}.
This is inefficient for HPC systems where files can be gigabytes or more in
size. Our middleware avoids this data migration by interpreting directory
entries as symbolic links: each directory entry (the name created by the
application) has a physical pathname that points to a file
with the actual file content stored in the underlying cluster file system.
This representation of directory entries is enabled through the use
of a novel on-disk metadata representation called \tfs \cite{TableFS},
based on a log-structure merge tree (LSM-tree) data structure \cite{ONeil1996}.
We use the \tfs approach to compact all file system metadata
(including directories, i-node attributes) and small files,
into flat files sorted on a unique key.
This organization facilitates high-speed metadata creation, lookups and scans.

Effectively integrating \tfs metadata store with the distributed indexing
technique requires several optimizations including cross-server split operations
with minimum data migration, and decoupling data and metadata paths.
To demonstrate the feasibility of our approach,
we implemented a prototype middleware layer called \sys and evaluated it
on an existing Panasas PanFS deployment \cite{PanFS} that has 5 shelves consisting of 50 nodes.
Our results show promising scalability and performance:
\sys layering on top of PanFS (\psys) was more than 10$\times$ faster than the original PanFS
in metadata intensive workloads, and performs comparably in data-intensive workloads.
