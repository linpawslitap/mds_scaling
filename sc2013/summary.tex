\section{Conclusion}
\label{conclusion}

Many HPC cluster file systems lack a general-purpose scalable metadata service
that distributes both namespace and directories. This paper presents an approach
that allows {\it existing} file systems to deliver scalable and parallel 
metadata performance. The key idea is to re-use a cluster file system's
scalable data path to provide concurrent access on the metadata path. 
Our experimental prototype has demonstrated a {\it 10X} improvement in the
metadata performance of a production cluster file system, Panasas's PanFS. 

This paper makes three contributions.
The first contribution is an efficient combination of scale-out indexing 
technique (GIGA+) with a scale-up metadata representation (TableFS) to enhance 
the scalability and performance of metadata operations.
The second contribution is the ease of deployment: our system can be layered 
on a cluster file system without any changes to the file system. This layering 
also facilitates easy federation of different file systems through a middleware 
library, an approach that is popular with other HPC systems such as MPI-IO and
PLFS.
The third contribution is the portable design that can use any existing file
system deployments without any configuration changes (but with different fault
tolerance assumptions) to the file system or the systems software on compute
nodes. 

